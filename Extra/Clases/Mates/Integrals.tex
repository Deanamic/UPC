% Created 2018-11-23 Fri 22:02
% Intended LaTeX compiler: pdflatex
\documentclass[11pt]{article}
\usepackage[utf8]{inputenc}
\usepackage[T1]{fontenc}
\usepackage{graphicx}
\usepackage{grffile}
\usepackage{longtable}
\usepackage{wrapfig}
\usepackage{rotating}
\usepackage[normalem]{ulem}
\usepackage{amsmath}
\usepackage{textcomp}
\usepackage{amssymb}
\usepackage{capt-of}
\usepackage{hyperref}
\usepackage{tikz}
\usepackage[margin=2cm]{geometry}
\date{February 3, 2018}
\title{Integrals}
\author{Dean Zhu}
\begin{document}
\maketitle
\section{Antiderivatives}
\label{sec:orgbf18620}

\subsection{A remark about differentiation.}
\label{sec:orgac82feb}

Suppose we want to calculate the speed of change of a function \emph{f(x)}. Let's draw
a parallel to physics, we have a particle which is moving and we are given it's
position as a function of time, that is, x(t). If we were to estimate it's speed
\emph{v} at time \(t_{0}\), then we would simply measure for a few seconds and see how much
its position has changed. So: \[ v(t_{0}) \approx \frac{\Delta x}{\Delta t} =
\frac{x_{1} - x_{0}}{t_{1} - t_{0}} \]

Now imagine if we wanted to give a more accurate measure at the instant, then
obviously instead of measuring for a few seconds, measuring only one second
would give us a better estimate, in fact, the shorter the period the more
accurate will be our estimation. (As long as we can measure it's position
accurately which is always a given in mathematics).

If we right this down a little more formally we get that: 

\begin{equation}
v(t_0) =
 \lim_{h\rightarrow 0} \frac{x(t_{0}+h) - x(t_0)}{(t_{0} + h) - t_{0}} = \lim_{h\rightarrow 0}
 \frac{x(t_{0}+h) - x(t_0)}{h}
\end{equation}

And if we consider now our function of position simply as \emph{f(x)}, we can see
that the speed of it's change it's simply is derivative which we will right as
\(\frac{dx}{dy}\). Note that \emph{d} stands for differential which
simply is a infinitely small difference, hence \(\frac{dx}{dy}\) is simply a
small change in \emph{x} divided by a small change in \emph{y}, essentially the same as
(1)

\subsection{Introduction to antiderivatives.}
\label{sec:org2a549b4}
Now suppose that we have the inverse problem in our physics example, we are
given the speed of some object and we are asked for it's postion. As we deduced
above we know the relationship between the function of position and it's speed
of change, which is the \emph{derivative}, hence it is natural to think that if we
had an operation which reversed this operation we would be able to solve the
proposed problem.

But as we know, the derivative of a constant function is 0, i.e. the speed of
change of a constant function is 0. Hence if we are given: 
\[ 
f(x) = c, \quad f'(x) = 0 
\] 

It becomes virtually imposible to know which constant function gave
as this derivative of value 0, so if there were an operation to reverse the
derivative it would be up to a constant value, note that if we were given the
value of f(x) at some point we would be able to determine the value of this
constant. This exact operation is called the integral or antiderivative, we will
call the integral of \emph{f(x)} a primitive.

\subsection{Geometric Interpretation}
\label{sec:orgec7a522}
We know that we can understand the derivative as a tangent to the function at a
certain point, we would also like to give an interpretation to the integral. 

Let's go back to our example, we defined an estimation of speed as the change in
position divided by the change in time, and we observed that the smaller the
timeframe of our measurements the more accurate would this estimation be. Let's
try and build the antiderivative in the same way.
If we are given the function of speed and some time and were asked to calculate
the position of the object relative to it's starting position, a very naive
estimation would simply be:
\[
x \approx v(0)*t_{0} \text{ or } x \approx v(t)*t_{0}
\]
That is, we suppose that during all the period the object moved at the same
speed as its initial one or it's final one. If we were asked to refine this estimation we could
for instance divide the period in two halves and say that: \[
x \approx v(0)\cdot\frac{t_{0}}{2} + v(\frac{t_{0}}{2})\cdot\frac{t_{0}}{2}
\]
We assume that at the first half it moved at the same speed as the initial one,
and at the latter half it moved at the same speed as its velocity at time
\(\frac{t_{0}}{2}\).
If we were to do this process n times we would obtain the following estimation:
\begin{equation}
x \approx \sum_{i = 0}^{n} v\bigg(i\cdot\frac{t_{0}}{n}\bigg) \bigg((i+1)\cdot\frac{t_{0}}{n} - i
\frac{t_{0}}{n}\bigg) = \sum_{i = 0}^{n} v\bigg(i\cdot\frac{t_{0}}{n}\bigg)\cdot\bigg(\frac{t_{0}}{n}\bigg)
\end{equation}
We divide the time in \emph{n} equal segments and the i-th segment starts at
\(i\cdot\frac{t_{0}}{n}\) and ends at \((i+1)\cdot\frac{t_{0}}{n}\).\\
Now if we were asked to give a more precise answer we could do the same as we
did with the derivative and assert that:
\begin{equation}
x = \lim_{n\rightarrow \infty} \sum_{i = 0}^{n} v\bigg(i\cdot\frac{t_{0}}{n}\bigg)\cdot\bigg(\frac{t_{0}}{n}\bigg)
\end{equation}
Note that \(\frac{t_{0}}{n}\) becomes infinitely small, in essence it becomes
a time differential hence we can write:
\begin{equation}
x = \lim_{n\rightarrow \infty} \sum_{i = 0}^{n} v(\frac{t_{0}}{n})dt
\end{equation}
That is, the integral is simply an infinite sum of the function multiplied by a
differential and thus:

\begin{equation}
x = \lim_{n\rightarrow \infty} \sum_{i = 0}^{n} v(i \cdot \frac{t_{0}}{n})dt = \int_{0}^{t_{0}} v(t) dt
\end{equation}

Now lets give this a geometrical meaning, consider the graph obtained by
plotting \emph{v(t)}, the first estimation simply consideted a \(v(0) \times t_{0}\)
rectangle as an estimation for \emph{x}. The second one considered two rectangles of
base \(\frac{t_{0}}{2}\) and height \(v(0) \text{ and } v(\frac{t_0}{2})\)
respectively. 

Now if we go to expression (2), we are expressing \emph{x} as the sum of \emph{n}
rectangles of base \(\frac{t_0}{n}\) and the height of the i-th rectangle being
\(v(i\cdot\frac{t_{0}}{n})\).

If we now look at the expression (5), we can see that the real value of the
integral is simply the sum of infinitely many rectangles of a base of length
\(dt\) (recall that dt is simply an infinitely small change of t) and its height begin our function \emph{v(t)} evaluated at the start of each
integral. That is, the integral is simply the are under the curve of our
function, this specific approximation using rectangles is known as the \textbf{Riemman Sum}.
\end{document}