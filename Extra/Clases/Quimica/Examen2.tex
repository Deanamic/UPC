% Created 2018-02-06 Tue 23:46
% Intended LaTeX compiler: pdflatex
\documentclass[11pt]{article}
\usepackage[utf8]{inputenc}
\usepackage[T1]{fontenc}
\usepackage{graphicx}
\usepackage{grffile}
\usepackage{longtable}
\usepackage{wrapfig}
\usepackage{rotating}
\usepackage[normalem]{ulem}
\usepackage{amsmath}
\usepackage{textcomp}
\usepackage{amssymb}
\usepackage{capt-of}
\usepackage{hyperref}
\usepackage[margin=2.5cm]{geometry}
\usepackage{titlesec}
\titleformat*{\section}{\bfseries}
\usepackage{xfrac}
\usepackage{enumitem}
\usepackage{titling}
\setlength{\droptitle}{-3cm}
\date{19 Febrero, 2018}
\title{Examen Química}
\author{Dean Zhu}
\begin{document}

\maketitle
\thispagestyle{empty}
El examen consta de 3 preguntas y se pueden obtener hasta 10.5 puntos sobre 10. En el ejercicio 2 los apartados no son dependientes de resultados anteriores ni están ordenados por dificultad. 
\section{[2.5pt]}
\label{sec:org220dcd1}
En el laboratorio tenemos una disolución de alcohol etílico 93\% en masa, queremos obtener una disolución de 500ml a 0.5 M. Calcula cuántos mililitros necesitamos de la disolución original, y cuantos mililitros de agua destilada harán falta. \\
Datos: Densidad del alcohol etílico = \(789kg/m^{3}\), \(\text{C} = 12g/mol \text{, O} = 16g/mol \text{, H} = 1g/mol \).

\section{[6pt]}
\label{sec:org466d254}
Nos encontramos en el laboratorio frente a dos disoluciones de ácido clorhídrico, una a 0.4M y la otra a 1M. Queremos distintas disoluciones para realizar experimentos:
\begin{enumerate}[label=\Alph*)]
\item Indica la disolución que utilizas y el volumen necesario para:

    \begin{itemize}
    \item {[1pt]} Obtener una disolución de 200ml a 0.2 M
    \item {[1pt]} Obtener una disolución de 100ml a 0.7 M
    \end{itemize}
  \item Después de medir el volumen de la disolución 1M vemos que solo tenemos 50ml de disolución. \\
    \textbf{Teniendo esto en cuenta}, queremos obtener ahora una disolución de 100ml a 0.6 M
    \begin{itemize}
    \item {[1pt]} Es posible obtener esta disolución utilizando solamente la disolución 0.4M, y solamente la disolución 1M? Justifica ambas respuestas.
    \item {[0.5pt]} Cuántos moles de HCl hay en los 50ml de la disolución 1M. Cuántos moles de HCl habrá en la disolución final?
    \item {[1.5pt]} Calcula el volumen de \textbf{agua destilada}, \textbf{disolución 1M} y \textbf{disolución 0.4M} necesarios.  \\
    \end{itemize}
  \item {[1pt]} Qué disolución tiene mayor molaridad la de 0.4M o la de 1M si la enrasamos hasta los 150ml. Ten en cuenta que solo tenemos 50ml de la disolución 1M \\
      Datos: \(\text{H} = 1g/mol \text{, Cl} = 35.5g/mol \).
  \end{enumerate}
  \section{[2pt]}
  Sabemos que el Aluminio en determinadas condiciones se oxida.
  \[
    \text{Al}_{\text{(s)}} + \text{O}_{\text{2(g)}} \rightarrow \text{Al}_{2}\text{O}_{\text{3(s)}}
  \]
  \begin{itemize}
  \item Iguala la reacción
    \item Si reaccionan 5g de oxigeno, Cuánta cantidad de óxido de aluminio se produce?
    \end{itemize}
          Datos: \(\text{O} = 16g/mol \text{, Al} = 27g/mol \).



\end{document}