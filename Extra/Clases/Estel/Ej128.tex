% Created 2017-12-13 Wed 21:30
% Intended LaTeX compiler: pdflatex
\documentclass[11pt]{article}
\usepackage[utf8]{inputenc}
\usepackage[T1]{fontenc}
\usepackage{graphicx}
\usepackage{grffile}
\usepackage{longtable}
\usepackage{wrapfig}
\usepackage{rotating}
\usepackage[normalem]{ulem}
\usepackage{amsmath}
\usepackage{textcomp}
\usepackage{amssymb}
\usepackage{capt-of}
\usepackage{hyperref}
\usepackage{tikz}
\usepackage[margin=2cm]{geometry}
\date{\today}
\title{}
\hypersetup{
 pdfauthor={},
 pdftitle={},
 pdfkeywords={},
 pdfsubject={},
 pdfcreator={Emacs 25.3.1 (Org mode 9.1.3)}, 
 pdflang={English}}
\begin{document}


\section{Ejercicio 128}
\label{sec:org10c8875}
Halla el area de regio limitada por la grafica de la funcion f(x) = \(\sqrt{x}\), el eje Y (x = 0) y la recta y = 4 (la llamare g(x), por tanto g(x) = 4)

\subsection{Grafica}
\label{sec:org0bfa64e}
Para hacer la grafica necesitamos algun punto importante, 
\subsubsection{f(x) con eje Y:}
\label{sec:org9850d85}
Como el eje Y es la recta vertical que pasa por el origen, sabemos que f(0) cortara con la recta. \\
x\(_{\text{1}}\) = \(\sqrt{0}\) = 0
\subsubsection{f(x) con g(x):}
\label{sec:org5eb5d2c}
\[ f(x) = 4 \iff \sqrt{x} = 4 \iff \lvert x \rvert = 16 \implies x = \pm 16 \]
Como hemos elevado al cuadrado, la solucion negativa no es possible (ademas \(\sqrt{x}\) cuando x es negativo no existe) \\
x\(_{\text{2}}\) = 16
\subsubsection{eje Y con g(x):}
\label{sec:orgdebb2e4}
Es obvio que corta cuando x = 0.
x\(_{\text{3}}\) = x\(_{\text{1}}\) = 0;

\begin{tikzpicture}[xscale=0.5]
\draw[->] (-1,0) -- (18,0) node[right] {$x$};
\draw[->] (0,-1) -- (0,5) node[above] {$y$};
\draw[blue, thick, domain=0:18, samples = 100] plot (\x, {sqrt(\x)}) node[right] {$f(x)$};;
\draw[red, thick, domain=0:18] plot (\x, {4}) node[right] {$g(x)$};
\draw[green, thick] (0,5) -- (0,0) node[below] {0}; ;
\draw[orange, thin] (16,4) -- (16,0)node[below] {16};
\end{tikzpicture}


Entonces ahora queremos calcular el area encerrada por las tres funciones:
\subsection{Resolucion}
\label{sec:org10228ca}
\[ Area = \int_{0}^{16} g(x) - f(x)  \]
Si nos fijamos, el area de g(x) es simplemente un rectangulo, de hecho Area es la diferencia entre el Area debajo g(x), que llamaremos A\(_{\text{1}}\). Y el area debajo de f(x) que llamaremos A\(_{\text{2}}\).\\
Entonces:
\[ Area = A_1 - A_2 = b*h - \int_{0}^{16} f(x) = 64 - \bigg[\frac{2*x^{\frac{3}{2}}}{3}\bigg]_{0}^{16} = 64 - \frac{2*16^{\frac{3}{2}}}{3} - 0 = 21.333\]
\end{document}